\section{The Lexiconinomicon} %: Things to Talk About

\subsection{Language in D\&D}
\vspace*{-8pt}
\quot{"Does anyone speak `Roper'? Anyone?"}

The default languages of Dungeons and Dragons (Goblin, Halfling, Giant, Common) assumes a certain level of racial tribalism, where a village is normally expected to be a "Halfling village" or a "Gnoll village," and that was the presentation of the D\&D world -- in 1977. The AD\&D Monster Manual talked about villages of Orcs or Goblins, and you could seriously count on your fingers the number of races that shared living space, and many of those "races" were just leveled versions of normal humanoids (flinds and lizardkings, for example).Thirty years have come and gone since those bad old days, and the modern presentation is much less "genetically isolated tribes" and much more "mixed species regions." Today when an enemy village is written up it has bugbears and orcs, and grimlocks and all kinds of crazy crap in it.

But the languages haven't changed, even though the presented social setup no longer supports that paradigm. A child grows up speaking whatever languages they happen to be exposed to, so when the Orcs were living on their own it was pretty reasonable for the language spoken by Orcs to be a different one from the other tribes and to be identified simply as "The Language that Orcs speak" or simply "Orc." But if any more complicated social system or demographic distribution is posited, that no longer works at all.

\subsubsection{Regional Languages}

The attempt to put Regional Languages into the mix has been a dismal failure. As anyone who has attempted to follow the Forgotten Realms language "system" can attest, that's something that you really have to put up the little finger quotes when you talk about it. A Regional Language is just a tribal language that at some point in time became influential enough that everyone in a region adopted it. That means that a Regional Language actually is "Orcish" -- it just means that the Orcs of that region kicked enough booty that everyone ended up learning Orc, and then in subsequent generations everyone spoke Orc and didn't even think that was weird. Heck, there might not be any Orcs in the area anymore. But everyone in that area will still speak Orc.

\subsubsection{Pidgins: Common and Undercommon}
\vspace*{-8pt}
\quot{"Orc take sword. I own sword. I tell you. I want sword. Orc give sword. I get sword. You tell orc."}

Common is not technically a language, it's a linguistic construct called a Pidgin. A Pidgin is a linguistic amalgamation that combines elements from several languages and has an extremely simple grammatical structure with no iterative capability. Whoa that's a lot of six-dollar words! The point is that all this stuff with relative clauses and structured inheritance that makes D\&D rulebooks read like a legal document is completely absent from a pidgin.

Pidgins form when people from different groups come together for trading purposes. So in the Underdark, Pidgin is pretty much just an extremely simplified version of the Drow version of Elvish. They trade with everyone, and speak in "Tourist Speak" where they speak very loudly and slowly in Drow and everyone has pretty much figured out what that means. Above ground, Common is mostly composed of the Halfling language, with a few loan words from other cultures thrown in.

The only reason that Common and Undercommon stay relatively static in D\&D is because the people actually doing the trading are crazy long lived and do the trading everywhere. The big traders in the D\&D economy are not those stupid caravans who wander around full of swag. No, it's Wizards and Outsiders who teleport expensive and wondrous stuff all over the planet. The reason why you can get a cup of coffee or a bolt of silk in your otherwise European villa is because people with teleportation are moving goods all over the place. So people who speak Common actually do share a common trade language with people clear on the other side of the planet. And they might not even know who the wizards in question are.

Interesting side note: People who grow up speaking a Pidgin as their only language actually speak a Creole, which is a real iterative language just like any other that is made out of the words of the original Pidgin. Human cultures in D\&D apparently default to Common as their primary language. That means that humans presumably speak Common as a language rather than as a pidgin. So the Wizards and Shadow Caravaners come to Human settlements from time to time and regard human speech as being filled with vulgar crazy-talk. The words are all there, but they have extra prepositions and jumble all the thoughts into single sentences.

\subsubsection{Language System I -- High Fantasy}

In the true High Fantasy setting, there are three languages on your continent, and no "Speak Language" skill.

Firstly, there is Common, which is what everyone speaks. Maybe people from far away speak a foreign, incomprehensible tongue, but it's foreign and incomprehensible and your characters don't speak it just because you an Intelligence of 12.

Secondly, there is The Old Tongue, which isn't spoken much, but is used in ancient writings and prophecies and such not. You can't have a speak language for this, to read it (or understand it), you need Decipher Script. This is what Decipher Script is for, since ancient script is generally in The Old Tongue. If you are a big bad ass elf you show off your many ranks in Decipher Script by peppering your speech with Old Tongue terms. If anyone asks, The Old Tongue is so complicated, full of subtle meaning and generally awesome that it can never be used for reliable communication.

Thirdly, there is The Dark Tongue, which is just like The Old Tongue from a game mechanical standpoint. To speak The Dark Tongue, you take The Old Tongue, change every other vowel into a hard consonant (a$\rightarrow$k, e$\rightarrow$t, i$\rightarrow$p, o$\rightarrow$g, u$\rightarrow$ch, y$\rightarrow$q), and all of the pauses (') become glottal stops (`). If you are a member of the evil political party, you pepper your speech with Dark Tongue words and phrases to prove how cool you are.
Ex.: L'rihylya'anyur cescelenti $\rightarrow$ L`rphylqa`knychr cesctlentp

The High Fantasy language system is about what you get from books like Shanarra or the Wheel of Endlessness cycle. It's also really easy, which is why it is in use by lazy authors. It also has the advantage that the Decipher Script skill has an obvious and explicable use (which let's face it: in standard D\&D it does not have, even deciphering magical writing is a Spellcraft check). People pretty much talk in English except when it's plot important that they be incomprehensible and everyone knows where everything stands. It's even less realistic than the basic rules, but it's closer to a lot of the important source material.

\subsubsection{Language System II -- Remotely Realistic}

Each major cultural group (e.g. Europe, China, India) has a language which they will call, more or less, "Classical" (e.g. Latin, Classical Han Chinese, Sanskrit, respectively.) This is the language that people will use for writing, it is also the language of discourse for travelers and the like. Classical is the only language with a meaningful written form, although you might find some scribbled notes or poems (e.g. the Golliard Poems, found at Carmina Burana) in a local, or vulgar, language. There are also pre-classical writings (e.g. Greek, Cuneiform, not to mention Old Slaan and Aboleth) which you will need Decipher Script to read.

You may or may not have mystical languages (Terran, Aquan, Celestial, etc.), if you do, it might be a good idea to have one of those serve in place of classical for one major cultural grouping or another. To save yourself trouble, assume that your world contained four great civilizations -- Northern, Southern, Eastern and Western. Each of these civilizations left behind a classical language, which is used for academic and administrative discourse in that region.

In this model, there is no "common" that is spoken by commoners. The tongue of the ancient Dwarvish Empire will be spoken by everyone in the Northern countries who is educated, but the uneducated commoners will speak all kinds of crazy local tongues (Wenn, Lapp, Prussian, etc.) and you may well have to turn to magical translation or local educated characters (such as the town wizard or a local aristocrat) in order to get your point across to the Plebes. This closely approximates the position that Latin had in medieval Europe or the position that Han Chinese had in medieval China.

\subsection{Spellbooks}
\vspace*{-8pt}
\quot{"Warning: may contain Explosive Runes."}

Long ago, a spellbook was an actual magical object. Magic Users could pop their book open, rip pages out and blast the contents out as magic scrolls. Their very method of spell preparation was to open this pile of dangerous magic items and concentrate on creating copies of the scrolls in their minds to be released later on as powerful magic. For those of you who are new to 3rd edition, that statement seems pretty weird, because it doesn't work that way at all anymore.

\subsubsection{Using Other Peoples' Spellbook}

You can pick up some other guy's spellbook and prepare spells out of them once you've deciphered the spell in question. The DC isn't even hard -- it's only 15+Spell Level (and you can take 10), so a high level character can't even fail. And by a "high level character," I actually mean a first level character if he has an Int Bonus of at least +2, which he does. With the extreme ease of using other peoples' spellbooks, one is tempted to ask why anyone ever makes a full scale copy of a spell -- that's crazy expensive.

The answer, of course, is that they don't. In reality all spells copied by powerful wizards are created with the secret page spell. That spell allows you to "hide" whatever is on a page with any writing you want -- even (specifically) spells. So the "fake page" is actually part of a spellbook, and the "real page" is probably just doodles of horses or tallies of wins and losses in Backgammon. Secret Page comes online for Wizards at level 5, so any Wizard of even modest power should be able to construct spellbooks in hours rather than days for zero gold pieces.

If you want, you can "master" another wizard's spellbook, at which point everything in it becomes just like you wrote it. The DC is 25 + Spell level

\subsubsection{Getting by Without a Spellbook}

People often assume that wizards carry their spellbooks with them at all times and that taking these books away from them will cripple their character beyond redemption. For low level adventuring wizards, this is essentially true. But for high level Wizards and wizards who don't adventure, nothing could be further from the truth.

Copies of spellbooks are astoundingly expensive -- but once characters enter the fabricate or wish economies of the upper levels, that cost is either meaningless or can be bypassed entirely (thanks secret page). Any decently high level wizard may well have dozens or hundreds of copies of that precious manuscript. That's why characters with a Wizard Mentor can copy spells for free -- the high level wizard literally just hands them a copy of any spell they figure out how to master.

\subsubsection{Arcane Magical Writings}

As written, a Wizard can learn a spell from any spellbook page or scroll she has deciphered. Deciphering a page or scroll is a spellcraft check that, among other things, tells you whether it is arcane or divine. That means that under the rules as written, a Wizard can take Cleric Scrolls and copy them into her spellbook and then they become Wizard spells of the same level. Honestly\ldots\  most DMs will not let you do that even though the PHB is extremely specific that that is exactly what you can do. But if it's really important to you to learn Cleric spells, you still can.

Many DMs put in the additional restriction that to learn a spell it must be Arcane, or even that it must be a Sor/Wiz spell. That's actually fine, because the world of D\&D includes Nagas, who cast Cleric spells as Sorcerer spells. They can make scrolls (or you can make a scroll with a Naga), and then you can learn those precious Cleric spells if you really care. Chances are, though, that you don't care. Clerics are much better than Wizards in every single aspect of their characters except in their spell-list. And while there are certainly some gems on the Cleric list as far as spells go, chances are if you wanted to build a character who casts those spells you'd be better off having been a Cleric in the first place. Have better hit points, Saves, and BAB. So while learning Cleric spells is probably a pretty stupid goal, it is definitely achievable no matter how strict your DM is.
